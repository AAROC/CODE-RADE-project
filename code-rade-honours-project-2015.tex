% Unofficial ERC Starting Grant LaTeX template
% Source: http://www.arj.no/2013/02/03/erc-stg-latex/
%
\documentclass[oneside, a4paper, onecolumn, 11pt]{article}

%%% PACKAGES %%%

\usepackage[left=2cm,top=2cm,bottom=1.5cm,right=2cm]{geometry}
\usepackage{hyperref}
\usepackage[utf8]{inputenc}
\usepackage{graphicx} 		% Add graphics capabilities
\usepackage{amsmath}  		% Better maths support
\usepackage{natbib}	% bibliography style
\setlength{\bibsep}{0.0pt}
\usepackage{eurosym}

% To fix list things:
\usepackage{enumitem}
\setitemize{noitemsep,topsep=0pt,parsep=0pt,partopsep=0pt,leftmargin=*}
\usepackage{amssymb}
\renewcommand{\labelitemi}{\tiny$\blacksquare$}

\usepackage{nopageno}
\usepackage{enumitem}

\usepackage{fancyhdr}
\pagestyle{fancy}
\renewcommand{\headrulewidth}{0pt} % Remove line at top

\lhead{\emph{Name}}
\chead{Part B1}
\rhead{ACRO}
\cfoot{\thepage}

\newenvironment{itemize*}%
  {\begin{itemize}%
    \setlength{\itemsep}{0pt}%
    \setlength{\parskip}{0pt}}%
  {\end{itemize}}

\usepackage{enumitem}

%%% COMMANDS %%%

% Define the title, author and date of the document.
\title{CODE-RADE:\\ }
\author{Bruce Becker\\ CSIR Meraka Institute}

%%% BODY OF THE DOCUMENT: %%%
\begin{document}

\noindent
University of the Witwatersrand

\vfill

\begin{center}
\large{\textbf{Wits University Honours Course\\
Research proposal }
}
\vfill

\LARGE{\textbf{Continuous Delivery of Research Applications in a Distributed Environment }}

\vfill

\LARGE{\textbf{CODE-RADE}}

\vfill

\end{center}

\vfill

\begin{itemize}
\item Principal investigator (PI): Bruce Becker (CSIR Meraka Institute)
\item Host institution: Wits University
\item Full title: Continuous Delivery of Research Applications in Distributed Environment
\item Proposal short name: CODE-RADE
\item Proposal duration: 10 months
\end{itemize}

%\maketitle

\vfill

%%% Probolem statement
\noindent
\section{Problem Statement}
One of the aims of e-Infrastructures is to provide easy access to powerful computational and data platforms, to as many eligible users as possible. The South African National Grid (SAGrid), as part of the National Integrated Cyberinfrastructure System is no exception. While access to the users is being simplified greatly by the adoption of science gateways and identity federations, the community of application developers and technical support in scientific collaborations does not yet have an easy way to integrate these applications in the first place.

SAGrid has identified this as a gap in the services it provides and has developed a solution to the issue of easily integrating new applications into the infrastructure in a fast, flexible, distributed and reproducible way. Using existing tools and services, we have defined a simple set of tests which applications need to pass in order to be considered valid for the infrastructure. These can been encoded as automated tests using a continuous integration platform such as Jenkins. Interoperability between source code repositories, automated build systems, artifact creation and content delivery systems is crucial for the sustainability and uptake of the system.

The specific problems we aim to address in this limited-scope project are those of automation and distribution. Specifically

\begin{itemize}
	\item \bf{How far is it feasible to automate the build steps in order to produce high-quality, redistributable artifacts ?}
	\item How can Linux containers be used to distributed these artifacts to remote sites with the minimum intervention necessary by site administrators
\end{itemize}


\section{References}

NICIS - The NICIS is a framework for the development of an integrated system for cyberinfrastrucutre in South Africa, including pillars of data, compute and network infrastructure.

Jenkins - Jenkins is a continuous integration platform widely used to automate testing and integration of applications.




%%% Bibliography
\begin{small}
\bibliographystyle{chicago}  % ama, nar, alpha, plain, chicago, abbrv, siam
\bibliography{my-bib.bib}
\end{small}

\end{document}